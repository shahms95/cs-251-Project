\documentclass[a4paper,12pt]{article}
\usepackage{graphicx}
\begin{document}
\title{Box2D Final Version}
\author{Group 21 - Vulcans!}
\date{\today}
\maketitle
\begin{section}{Introduction}
This is our Box 2D Project Report for the complete project - Making a Rube Goldberg Machine. 
The link to the webpage of our project is (Insert link here)
\end{section}
\bigskip

\begin{section}{Contents}
Contents of this document include -
\begin{itemize}
\item
Introduction
\item
Contents
\item
Honor code
\item
Original Schematic of our Rube Goldberg machine and Final implementation of our Rube Goldberg machine.
\item
Sequence of events in the simulation
\item
Explanation of each component in detail
\begin{itemize}
\item
Newton's Pendulum
\item
Dominos
\item
And many more..
\end{itemize}
\item
Techniques (Introduced in this course) Used
\item
Physical concepts used
\item
Implementation Plan/Sequence
\item
Task Distribution
\item
Challenging Aspects
\item
References
\end{itemize}
\end{section}
\bigskip

\begin{section}{Honor Code}
We pledge on our honour that we have not given or received any unauthorized assistance in this assignment or any previous task.
\end{section}
\bigskip

\begin{section}{Schematic}
Original schematic is as follows \\
(Insert original Inkscape image here)
\\
After deeply considering various inputs that we got from our Instructor, TAs and the graders, we added several new components to our original design and a screenshot of our final machine is - \\
(Insert current image here)
\end{section}
\bigskip

\begin{section}{Components Used}
Moving - \\
\begin{enumerate}
\item Newton's Pendulum
\item Dominos
\item Moving Planks (Kickers?)
\item Balls
\item Conveyor Belt
\item Pulley
\item Cannons
\item Explosions
\end{enumerate}
Static - \\
\begin{enumerate}
\item Splitting chain of events into two
\item Slide
\item Planks
\end{enumerate}
\end{section}
\bigskip

\begin{section}{Technical tools used in making this project (and its documentation)}
\begin{itemize}
\item Box2D
\item Latex, Bibtex
\item Git - Source Code Versioning
\item Doxygen
\item Makefiles
\item Code Profiling and Optimization
\item Inkscape
\item HTML, CSS
\item Gnuplot
\end{itemize}
\end{section}
\bigskip

\begin{section}{Physical Aspects of the elements in the project}
\begin{itemize}
\item Pendulum
\item Coefficient of Restitution
\item Friction
\item Energy Conservation
\item Rolling of Spheres
\item Projectile Motion
\end{itemize}
\end{section}
\bigskip

\begin{section}{Implementation Plan}
\begin{itemize}
\item
Phase 1:
We will start from the bottom up first implementing the bottom-most pendulum and dominoes.
\item
Phase 2: The staggered Dominoes will be added along with the towers on which they stand.
\item
Phase 3:
The platforms will be added along with the hinged elements.
Testing
\item
Phase 4:
Balls will be added on to the platforms
\item
Phase 5:
Pulley system will be created along with the platform guides for the final big ball.
\item
Phase 6:
The final Balls will be added.
\item
Phase 7:
Testing of the whole along with optimization to improve the toppling effect, and or some minor design changes to achieve some more dramatic destruction.
\end{itemize}
\end{section}
\bigskip

\begin{section}{Task Distribution}
\begin{enumerate}
\item Anand Dhoot - Phase 3,6 
\item Maulik Shah - Phase 1,4
\item Anchit Gupta - Phase 2,5
\end{enumerate}
\end{section}
\bigskip

\begin{section}{Challenging Aspects}
\begin{itemize}
\item Conveyor Belt
\item Explosions
\item Cannon-balls meeting at the same point in the air
\item Precise positioning of all elements and testing
\end{itemize}
\end{section}
\bigskip

\begin{section}{References}

\end{section}
\bigskip

\end{document}
